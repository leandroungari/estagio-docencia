%%%%%%%%%%%%%%%%%%%%%%%%%%%%%%%%%%%%%%%%%
% Beamer Presentation
% LaTeX Template
% Version 1.0 (10/11/12)
%
% This template has been downloaded from:
% http://www.LaTeXTemplates.com
%
% License:
% CC BY-NC-SA 3.0 (http://creativecommons.org/licenses/by-nc-sa/3.0/)
%
%%%%%%%%%%%%%%%%%%%%%%%%%%%%%%%%%%%%%%%%%

%----------------------------------------------------------------------------------------
%	PACKAGES AND THEMES
%----------------------------------------------------------------------------------------

\documentclass{beamer}

\mode<presentation> {

% The Beamer class comes with a number of default slide themes
% which change the colors and layouts of slides. Below this is a list
% of all the themes, uncomment each in turn to see what they look like.

%\usetheme{default}
%\usetheme{AnnArbor}
%\usetheme{Antibes}
%\usetheme{Bergen}
%\usetheme{Berkeley}
%\usetheme{Berlin}
%\usetheme{Boadilla}
%\usetheme{CambridgeUS}
%\usetheme{Copenhagen}
%\usetheme{Darmstadt}
%\usetheme{Dresden}
%\usetheme{Frankfurt}
%\usetheme{Goettingen}
%\usetheme{Hannover}
%\usetheme{Ilmenau}
%\usetheme{JuanLesPins}
\usetheme{Luebeck}
%\usetheme{Madrid}
%\usetheme{Malmoe}
%\usetheme{Marburg}
%\usetheme{Montpellier}
%\usetheme{PaloAlto}
%\usetheme{Pittsburgh}
%\usetheme{Rochester}
%\usetheme{Singapore}
%\usetheme{Szeged}
%\usetheme{Warsaw}

% As well as themes, the Beamer class has a number of color themes
% for any slide theme. Uncomment each of these in turn to see how it
% changes the colors of your current slide theme.

%\usecolortheme{albatross}
%\usecolortheme{beaver}
%\usecolortheme{beetle}
%\usecolortheme{crane}
%\usecolortheme{dolphin}
%\usecolortheme{dove}
%\usecolortheme{fly}
%\usecolortheme{lily}
%\usecolortheme{orchid}
%\usecolortheme{rose}
%\usecolortheme{seagull}
%\usecolortheme{seahorse}
%\usecolortheme{whale}
%\usecolortheme{wolverine}

%\setbeamertemplate{footline} % To remove the footer line in all slides uncomment this line
%\setbeamertemplate{footline}[page number] % To replace the footer line in all slides with a simple slide count uncomment this line


\makeatletter
\setbeamertemplate{footline}
{
  \leavevmode%
  \hbox{%
  \begin{beamercolorbox}[wd=.333333\paperwidth,ht=2.25ex,dp=1ex,center]{author in head/foot}%
    \usebeamerfont{author in head/foot}\insertshortauthor
  \end{beamercolorbox}%
  \begin{beamercolorbox}[wd=.333333\paperwidth,ht=2.25ex,dp=1ex,center]{title in head/foot}%
    \usebeamerfont{title in head/foot}\insertshorttitle
  \end{beamercolorbox}%
  \begin{beamercolorbox}[wd=.333333\paperwidth,ht=2.25ex,dp=1ex,right]{date in head/foot}%
    \usebeamerfont{date in head/foot}\insertshortdate{}\hspace*{2em}
    \insertframenumber{} / \inserttotalframenumber\hspace*{2ex} 
  \end{beamercolorbox}}%
  \vskip0pt%
}
\makeatother



%\setbeamertemplate{navigation symbols}{} % To remove the navigation symbols from the bottom of all slides uncomment this line
}

% Allows including images
\usepackage{graphicx}
% Allows the use of \toprule, \midrule and \bottomrule in tables
\usepackage{booktabs} 
% Allows portuguese language
\usepackage[brazil]{babel}
% Enables utf8 encoding
\usepackage[utf8]{inputenc}
% Enables some alignment
\usepackage{ragged2e}
% Enables hyperlinks
\usepackage{hyperref}
% Enables diagram
\usepackage{tikz}
% Enables subfigures
\usepackage{subfig}
% Allows uml diagrams
\usepackage{style/tikz-uml}


%%%%%%%%%%%%%
\usepackage{inconsolata}

\usepackage{color}

\definecolor{pblue}{rgb}{0.13,0.13,1}
\definecolor{pgreen}{rgb}{0,0.5,0}
\definecolor{pred}{rgb}{0.9,0,0}
\definecolor{pgrey}{rgb}{0.46,0.45,0.48}

\usepackage{listings}
\lstset{language=Java,
  showspaces=false,
  showtabs=false,
  breaklines=true,
  showstringspaces=false,
  breakatwhitespace=true,
  commentstyle=\color{pgreen},
  keywordstyle=\color{pblue},
  stringstyle=\color{pred},
  moredelim=[il][\textcolor{pgrey}]{$$},
  moredelim=[is][\textcolor{pgrey}]{\%\%}{\%\%},
  basicstyle=\footnotesize
}

%----------------------------------------------------------------------------------------
%	TITLE PAGE
%----------------------------------------------------------------------------------------

% The short title appears at the bottom of every slide, the full title is only on the title page
\title[Polimorfismo]
{Polimorfismo e Exemplos} 

% Your name
\author[Leandro U.C., Danilo M.E.]
{Leandro Ungari Cayres\inst{1}, Danilo Medeiros Eler\inst{2}} 

% Your institution as it will appear on the bottom of every slide, may be shorthand to save space
\institute[UNESP] 
{
% Your institution for the title page
Universidade Estadual Paulista \\ 
\medskip
% Your email address
\textit{leandro.ungari@unesp.br\inst{1}, danilo.eler@unesp.br\inst{2}} 
}
% Date, can be changed to a custom date
\date{\today} 

\begin{document}

\begin{frame}
% Print the title page as the first slide
\titlepage 
\end{frame}

\begin{frame}
\frametitle{Visão Geral} 
% Table of contents slide, comment this block out to remove it
\tableofcontents 
% Throughout your presentation, if you choose to use \section{} and \subsection{} commands, these will automatically be printed on this slide as an overview of your presentation
\end{frame}

%----------------------------------------------------------------------------------------
%	PRESENTATION SLIDES
%----------------------------------------------------------------------------------------
\section{Conceitos}

%----------------------------------------------------------------------------------------
%----------------------------------------------------------------------------------------
\subsection{Definição}

%----------------------------------------------------------------------------------------
\begin{frame}
\frametitle{Introdução}
\justifying
\quad Neste ponto da Orientação a Objetos, foram introduzidos dois pilares desse paradigma, o Encapsulamento e a Herança, enquanto o primeiro permite a implementação de componentes independentes, o segundo possibilita a reutilização hierárquica desses.
\end{frame}

%----------------------------------------------------------------------------------------
\begin{frame}
\frametitle{Polimorfismo}
\begin{block}{Definição}
\qquad Ter muitas formas. Em termos de programação, muitas formas significa que um único nome pode representar um código diferente, selecionado por algum elemento automático. Assim, o polimorfismo permite que um único nome expresse muitos comportamentos diferentes~\cite{sintes2002aprenda}.
\end{block}
\end{frame}

%----------------------------------------------------------------------------------------
\begin{frame}
\frametitle{Características}
O polimorfismo atende cada um dos objetivos da orientação a objetos, permitindo a elaboração de softwares com as seguintes características~\cite{sintes2002aprenda}:

\begin{itemize}
\item Natural\\{\footnotesize O polimorfismo permite que modele o mundo de forma mais natural, trabalhando em nível mais genérico e conceitual.}
\item Confiável\\{\footnotesize O polimorfismo resulta em código confiável pois simplifica os casos, eliminando situações especiais e isolando o código, assim reduz a chance de introduzir defeitos.}
\end{itemize}
\end{frame}

%----------------------------------------------------------------------------------------
\begin{frame}
\frametitle{Características}
\begin{itemize}
\item Reutilizável\\{\footnotesize O polimorfismo permite reaproveitar implementadas já realizadas, somente adequando conversa a interface utilizada, sempre precisar conhecer detalhes específicos.}
\item Manutenível\\{\footnotesize O polimorfismo resulta em menos código, por consequência menos código precisa ser mantido, ou seja, menos trabalho.}
\end{itemize}
\end{frame}

%----------------------------------------------------------------------------------------
\begin{frame}
\frametitle{Características}
\begin{itemize}
\item Extensível\\{\footnotesize O polimorfismo permite que novos tipos sejam adicionados sem afetar as demais partes do sistema, assim a integração é facilitada}.
\item Oportuno\\{\footnotesize O polimorfismo permite escrever menos código, e consequentemente, pode distribuí-lo mais cedo.}
\end{itemize}
\end{frame}

%----------------------------------------------------------------------------------------
%----------------------------------------------------------------------------------------
\subsection{Exemplo inicial}
%fazer exemplo de calculo de área de figuras geometricas

%----------------------------------------------------------------------------------------
\begin{frame}
\frametitle{Cálculo de Área}
\quad Uma aplicação está sendo desenvolvida para o cálculo da áreas, requerindo a necessidade implementação de algumas figuras geométricas. 

\quad Como podemos definir o cálculo da área para essas figuras?
\end{frame}

%----------------------------------------------------------------------------------------
\begin{frame}[fragile]
\frametitle{Cálculo de Área}
\justifying
Um diagrama de classe que projetaria esta aplicação.
\\~\\
\begin{center}
\resizebox{.3\textwidth}{!}{
\begin{tikzpicture}
\umlclass[x=0,y=0]{Shape}{...}{
	+ area(): double
}
\end{tikzpicture}
}
\end{center}
\end{frame}

%----------------------------------------------------------------------------------------
\begin{frame}[fragile]
\frametitle{Cálculo de Área}
\centering
\begin{tikzpicture}[scale=2]
    \tikzstyle{ann} = [draw=none,fill=none,right]
    \matrix[nodes={draw, ultra thick, fill=blue!20},
        row sep=0.3cm,column sep=0.5cm] {
    \node[rectangle] {$A = b.h$}; &
    \node[circle] {$A = \pi.r^2$}; & 
    \node[regular polygon,regular polygon sides=4] {$A = l^2$}; &
    \node[regular polygon,regular polygon sides=3, scale=.65] {$A = (b.h)/2$}; \\
};
\end{tikzpicture}
\end{frame}

%----------------------------------------------------------------------------------------
\begin{frame}[fragile]
\frametitle{Cálculo de Área}
\justifying
Um diagrama de classe que projetaria esta aplicação.
\\~\\
\begin{center}
\resizebox{\textwidth}{!}{
\begin{tikzpicture}
\umlclass[x=0,y=0]{Shape}{...}{
	\umlvirt{+ area(): double}
}
\umlclass[x=-5.5,y=-3]{Rectangle}{}{
	+ area(): double
}
\umlclass[x=-2,y=-3]{Circle}{}{
	+ area(): double
}
\umlclass[x=2,y=-3]{Square}{}{
	+ area(): double
}
\umlclass[x=5.5,y=-3]{Triangle}{}{
	+ area(): double
}
\umlimpl{Rectangle}{Shape}
\umlimpl{Circle}{Shape}
\umlimpl{Square}{Shape}
\umlimpl{Triangle}{Shape}
\end{tikzpicture}
}
\end{center}
\end{frame}

%----------------------------------------------------------------------------------------
\begin{frame}[fragile]
\frametitle{Cálculo de Área -- Implementação}
\begin{lstlisting}
public class Shape {
  ...
  public abstract double area();
}

public class Rectangle extends Shape {
  
  private double base;
  private double height;
  ...
  public abstract double area() {
    return this.base*this.height;
  }
}
\end{lstlisting}
\end{frame}

%----------------------------------------------------------------------------------------
\begin{frame}[fragile]
\frametitle{Cálculo de Área -- Implementação}
\begin{lstlisting}


public class Circle extends Shape {
  
  private double radius;
  ...
  public abstract double area() {
    return Math.PI*Math.pow(this.radius, 2);
  }
}

public class Square extends Shape {
  
  private double side;
  ...
\end{lstlisting}
\end{frame}

%----------------------------------------------------------------------------------------
\begin{frame}[fragile]
\frametitle{Cálculo de Área -- Implementação}
\begin{lstlisting}

  ...
  public abstract double area() {
    return Math.pow(this.side, 2);
  }
}

public class Triangle extends Shape {
  
  private double base;
  private double height;
  ...
  public abstract double area() {
    return this.base*this.height/2;
  }
}
\end{lstlisting}
\end{frame}

\section{Tipos de Polimorfismo}

%----------------------------------------------------------------------------------------
%----------------------------------------------------------------------------------------
\subsection*{}

%----------------------------------------------------------------------------------------
\begin{frame}
\frametitle{Categorias}
\justifying
\quad O Polimorfismo pode ser aplicado de diversos modos, conforme a necessidade de cada implementação~\cite{sintes2002aprenda}. 
A seguir são apresentados quatros principais categorias dessa abordagem:

\begin{itemize}
	\item Polimorfismo de Inclusão
	\item Polimorfismo Paramétrico
	\item Sobreposição
	\item Sobrecarga
\end{itemize}
\end{frame}

%----------------------------------------------------------------------------------------
%----------------------------------------------------------------------------------------
\subsection{Polimorfismo de Inclusão}

%----------------------------------------------------------------------------------------
\begin{frame}
\frametitle{Definição}
\justifying
\begin{block}{Conceito}
\qquad O polimorfismo de inclusão, também conhecido como polimorfismo puro, permite que você trate objetos relacionados genericamente.
\end{block}
\end{frame}

%----------------------------------------------------------------------------------------
\begin{frame}
\frametitle{Exemplo}
\centering
\resizebox{.8\textwidth}{!}{
\begin{tikzpicture}
\umlclass[x=0,y=0]{Vehicle}{...}{...}
\umlclass[x=-3,y=-3]{Motorcycle}{}{
	+ accelerate()
}
\umlclass[x=0,y=-3]{Car}{}{
	+ accelerate()
}
\umlclass[x=3,y=-3]{Truck}{}{
	+ accelerate()
}
\umlimpl{Motorcycle}{Vehicle}
\umlimpl{Car}{Vehicle}
\umlimpl{Truck}{Vehicle}
\end{tikzpicture}
}
\end{frame}

%----------------------------------------------------------------------------------------
\begin{frame}[fragile]
\frametitle{Exemplo}
\justifying
\begin{lstlisting}
public void accelerateVehicle(Motorcycle obj){
  obj.accelerate();
}

public void accelerateVehicle(Car obj){
  obj.accelerate();
}

public void accelerateVehicle(Truck obj){
  obj.accelerate();
}
\end{lstlisting}
\end{frame}

%----------------------------------------------------------------------------------------
\begin{frame}[fragile]
\frametitle{Exemplo}
\justifying
\quad Todos os métodos anteriores são equivalentes a esse:
\\~\\
\begin{lstlisting}
public void accelerateVehicle(Vehicle obj){
  obj.accelerate();
}
\end{lstlisting}
\end{frame}


%----------------------------------------------------------------------------------------
%----------------------------------------------------------------------------------------
\subsection{Polimorfismo Paramétrico}

%----------------------------------------------------------------------------------------
\begin{frame}
\frametitle{Definição}
\justifying
\begin{block}{Conceito}
\qquad O polimorfismo paramétrico que você crie métodos e tipos genéricos. 
Essa categoria pode ser aplicada em tipos e métodos, na primeira os tipos de parâmetros e retornos podem ser generalizados, enquanto no segundo, as referências de tipos dos métodos são omitidos até o momento da execução.
\end{block}
\end{frame}

%----------------------------------------------------------------------------------------
\begin{frame}[fragile]
\frametitle{Exemplo}
\justifying
\begin{lstlisting}
//Polimorfismo parametrico de Tipo
public class Queue<T> {
	
  void enqueue(T element) { ... }

  T dequeue() { ... }

  boolean isEmpty() { ... }

}
\end{lstlisting}
\end{frame}

%----------------------------------------------------------------------------------------
\begin{frame}[fragile]
\frametitle{Exemplo}
\justifying
\begin{lstlisting}
//Metodos de adicao de elementos
int add(int a, int b) { ... }
double add(double a, double b) { ... }
int[][] add(int[][] a, int[][] b) { ... }

//Polimorfismo parametrico de Metodo
T add(T a, T b) { ... }
\end{lstlisting}
\end{frame}

%----------------------------------------------------------------------------------------
%----------------------------------------------------------------------------------------
\subsection{Sobreposição}

%----------------------------------------------------------------------------------------
\begin{frame}
\frametitle{Definição}
\justifying
\begin{block}{Conceito}
\qquad O polimorfismo de sobreposição é um tipo importante, no qual o comportamento em uma hierarquia é sobreposto por um comportamento mais específico
\end{block}
\end{frame}

%----------------------------------------------------------------------------------------
\begin{frame}[fragile]
\frametitle{Exemplo}
\justifying
\quad Uma classe~\textbf{Geometric} é definida, da qual derivam outras duas classes~\textbf{Square} e~\textbf{Triangle}.
\begin{lstlisting}
public abstract class Geometric {
	
  public abstract double area();
}
\end{lstlisting}
\end{frame}

%----------------------------------------------------------------------------------------
\begin{frame}[fragile]
\frametitle{Exemplo}
\justifying
\begin{lstlisting}
@Override
public abstract class Square extends Geometric {
	
  public double area() {
    return this.side*this.side;
  }
}

@Override
public abstract class Triangle extends Geometric {
	
  public double area() {
    return this.base*this.height/2;
  }
}
\end{lstlisting}
\end{frame}

%----------------------------------------------------------------------------------------
%----------------------------------------------------------------------------------------
\subsection{Sobrecarga}

%----------------------------------------------------------------------------------------
\begin{frame}
\frametitle{Definição}
\justifying
\begin{block}{Conceito}
\qquad O polimorfismo de sobrecarga, também conhecido como polimorfismo~\emph{ad-hoc}, permite que o nome do método seja o mesmo para diferentes métodos, diferindo apenas na quantidade de parâmetros e os tipo deles.
\end{block}
\end{frame}

%----------------------------------------------------------------------------------------
\begin{frame}[fragile]
\frametitle{Exemplo}
\justifying
\begin{lstlisting}
public class MyOwnDB {
	
  //It returns all items
  public List<Item> search() { ... } 

  //It returns all items which are filtered by condition
  public List<Item> search(String condition) { ... } 
  
  //... and ordered by attribute
  public List<Item> search(String condition, String orderBy) { ... } 
  
  //It returns limited number of items which are filtered by condition and ordered by attribute
  public List<Item> search(String condition, String orderBy) { ... } 
}
\end{lstlisting}
\end{frame}
\section{Outros Exemplos}

%----------------------------------------------------------------------------------------
%----------------------------------------------------------------------------------------
\subsection*{}

%----------------------------------------------------------------------------------------
\begin{frame}
\centering
\Huge Outros Exemplos
\end{frame}

%----------------------------------------------------------------------------------------
%----------------------------------------------------------------------------------------
\subsection{Pagamento de Compras}

%----------------------------------------------------------------------------------------
\begin{frame}
\frametitle{Exemplo I}
\begin{columns}
\begin{column}{.5\textwidth}
\justifying
\quad Uma pequena empresa aceitava inicialmente somente pagamentos em dinheiro, desse modo, seu sistema de caixa fora projetado para tal propósito.
\end{column}
\begin{column}{.5\textwidth}
\begin{tikzpicture}
\node[inner sep=0pt] (cashier) at (-2,2)
    {\href{https://www.flaticon.com/authors/freepik}
		{\includegraphics[width=.2\columnwidth]{images/cashier.png}}
	};
\node[inner sep=0pt] (cash) at (0,0)
    {\href{https://www.flaticon.com/authors/dinosoftlabs}
		{\includegraphics[width=.2\columnwidth]{images/cash.png}}
	};
\node[inner sep=0pt] (check) at (2,-2)
    {\href{https://www.flaticon.com/authors/maxim-basinski}
		{\includegraphics[width=.2\columnwidth]{images/checked.png}}
	};
\draw[->,thick] 
	(cashier.south east) -- (cash.north west)
    node[midway,fill=white] {Pagamento};
\draw[->,thick] 
	(cash.south east) -- (check.north west)
    node[midway,fill=white] {Aprovação};
\end{tikzpicture}
\end{column}
\end{columns}
% paypal -> https://www.flaticon.com/authors/roundicons
% credit card -> https://www.flaticon.com/authors/smashicons
% money -> https://www.flaticon.com/authors/dinosoftlabs
%check -> https://www.flaticon.com/authors/maxim-basinski
%cashier -> https://www.flaticon.com/authors/freepik
\end{frame}

%----------------------------------------------------------------------------------------
\begin{frame}
\frametitle{Representação}
\resizebox{\textwidth}{!}{
\begin{tikzpicture}
\umlclass[x=0,y=0]{Purchase}{
	- value: double\\
	- closed: boolean \\
	- payment: CashPayment
}{
	+ finalize(): boolean\\
	+ setPayment(CashPayment c): void\\
	+ getTotal(): double
}
\umlclass[x=8,y=0]{CashPayment}{
	- value: double\\
	- isApproved: boolean
}{
	+ execute() : void\\
	+ isApproved() : boolean
}
\umlunicompo[geometry=-|-,mult1=1,mult2=1,pos1=0.2,pos2=2.8]
{Purchase}{CashPayment}
\end{tikzpicture}
}
\end{frame}

%----------------------------------------------------------------------------------------
\begin{frame}[fragile]
\frametitle{Implementação}
\begin{lstlisting}
public class Purchase {
  ...

  public boolean finalize() {
    ...
    //type CashPayment
    this.payment.execute();
  }
}

Purchase purchase = new Purchase();
CashPayment cash = new CashPayment(purchase.getTotal());

purchase.setPayment(cash);
purchase.finalize();
\end{lstlisting}
\end{frame}


%----------------------------------------------------------------------------------------
\begin{frame}
\frametitle{Exemplo I}
\quad Com o crescimento da empresa e também solicitação dos usuários, novas formas de pagamento devem ser adicionadas.
\begin{figure}
  \centering
  \subfloat{
  	\href{https://www.flaticon.com/authors/dinosoftlabs}
  	{\includegraphics[height=1.5cm,width=1.5cm]{images/cash.png}}
  }\hspace{1cm}
  \subfloat{
  	\href{https://www.flaticon.com/authors/smashicons}
  	{\includegraphics[height=1.5cm,width=1.5cm]{images/credit-card.png}}
  }\hspace{1cm}
  \subfloat{
  	\href{https://www.flaticon.com/authors/roundicons}
  	{\includegraphics[height=1.5cm,width=1.5cm]{images/paypal.png}}
  }
\label{fig:1}
\end{figure}
\quad Qual é a melhor estratégia a ser desenvolvida na implementação do sistema?
\end{frame}

%----------------------------------------------------------------------------------------
\begin{frame}
\frametitle{Opção I}
\centering
\begin{tikzpicture}
\node[inner sep=0pt] (cashier) at (-4,0)
    {\href{https://www.flaticon.com/authors/freepik}
		{\includegraphics[width=.12\textwidth]{images/cashier.png}}
	};
\node[inner sep=0pt] (cash) at (0,0)
    {\href{https://www.flaticon.com/authors/dinosoftlabs}
		{\includegraphics[width=.12\textwidth]{images/cash.png}}
	};
\node[inner sep=0pt] (check) at (4,0)
    {\href{https://www.flaticon.com/authors/maxim-basinski}
		{\includegraphics[width=.12\textwidth]{images/checked.png}}
	};
\draw[->,thick] 
	(cashier.east) -- (cash.west)
    node[midway,fill=white] [below] {Pagamento};
\draw[->,thick] 
	(cash.east) -- (check.west)
    node[midway,fill=white] [below] {Aprovação};
\end{tikzpicture}
\end{frame}

%----------------------------------------------------------------------------------------
\begin{frame}
\frametitle{Opção I}
\centering
\begin{tikzpicture}
\node[inner sep=0pt] (cashier0) at (-4,1)
    {\href{https://www.flaticon.com/authors/freepik}
		{\includegraphics[width=.1\textwidth]{images/cashier.png}}
	};
\node[inner sep=0pt] (cash0) at (0,1)
    {\href{https://www.flaticon.com/authors/dinosoftlabs}
		{\includegraphics[width=.1\textwidth]{images/cash.png}}
	};
\node[inner sep=0pt] (check0) at (4,1)
    {\href{https://www.flaticon.com/authors/maxim-basinski}
		{\includegraphics[width=.1\textwidth]{images/checked.png}}
	};
\draw[->,thick] 
	(cashier0.east) -- (cash0.west)
    node[midway,fill=white] [below] {Pagamento};
\draw[->,thick] 
	(cash0.east) -- (check0.west)
    node[midway,fill=white] [below] {Aprovação};
%%%%%%%%%%%%%%%
\node[inner sep=0pt] (cashier1) at (-4,-1)
    {\href{https://www.flaticon.com/authors/freepik}
		{\includegraphics[width=.1\textwidth]{images/cashier.png}}
	};
\node[inner sep=0pt] (cash1) at (0,-1)
    {\href{https://www.flaticon.com/authors/smashicons}
		{\includegraphics[width=.1\textwidth]{images/credit-card.png}}
	};
\node[inner sep=0pt] (check1) at (4,-1)
    {\href{https://www.flaticon.com/authors/maxim-basinski}
		{\includegraphics[width=.1\textwidth]{images/checked.png}}
	};
\draw[->,thick] 
	(cashier1.east) -- (cash1.west)
    node[midway,fill=white] [below] {Pagamento};
\draw[->,thick] 
	(cash1.east) -- (check1.west)
    node[midway,fill=white] [below] {Aprovação};
\end{tikzpicture}
\end{frame}

%----------------------------------------------------------------------------------------
\begin{frame}
\frametitle{Opção I}
\centering
\begin{tikzpicture}
\node[inner sep=0pt] (cashier0) at (-4,2)
    {\href{https://www.flaticon.com/authors/freepik}
		{\includegraphics[width=.08\textwidth]{images/cashier.png}}
	};
\node[inner sep=0pt] (cash0) at (0,2)
    {\href{https://www.flaticon.com/authors/dinosoftlabs}
		{\includegraphics[width=.08\textwidth]{images/cash.png}}
	};
\node[inner sep=0pt] (check0) at (4,2)
    {\href{https://www.flaticon.com/authors/maxim-basinski}
		{\includegraphics[width=.08\textwidth]{images/checked.png}}
	};
\draw[->,thick] 
	(cashier0.east) -- (cash0.west)
    node[midway,fill=white] [below] {Pagamento};
\draw[->,thick] 
	(cash0.east) -- (check0.west)
    node[midway,fill=white] [below] {Aprovação};
%%%%%%%%%%%%%%%
\node[inner sep=0pt] (cashier1) at (-4,0)
    {\href{https://www.flaticon.com/authors/freepik}
		{\includegraphics[width=.08\textwidth]{images/cashier.png}}
	};
\node[inner sep=0pt] (cash1) at (0,0)
    {\href{https://www.flaticon.com/authors/smashicons}
		{\includegraphics[width=.08\textwidth]{images/credit-card.png}}
	};
\node[inner sep=0pt] (check1) at (4,0)
    {\href{https://www.flaticon.com/authors/maxim-basinski}
		{\includegraphics[width=.08\textwidth]{images/checked.png}}
	};
\draw[->,thick] 
	(cashier1.east) -- (cash1.west)
    node[midway,fill=white] [below] {Pagamento};
\draw[->,thick] 
	(cash1.east) -- (check1.west)
    node[midway,fill=white] [below] {Aprovação};
%%%%%%%%%%%%%%%
\node[inner sep=0pt] (cashier2) at (-4,-2)
    {\href{https://www.flaticon.com/authors/freepik}
		{\includegraphics[width=.08\textwidth]{images/cashier.png}}
	};
\node[inner sep=0pt] (cash2) at (0,-2)
    {\href{https://www.flaticon.com/authors/roundicons}
		{\includegraphics[width=.08\textwidth]{images/paypal.png}}
	};
\node[inner sep=0pt] (check2) at (4,-2)
    {\href{https://www.flaticon.com/authors/maxim-basinski}
		{\includegraphics[width=.08\textwidth]{images/checked.png}}
	};
\draw[->,thick] 
	(cashier2.east) -- (cash2.west)
    node[midway,fill=white] [below] {Pagamento};
\draw[->,thick] 
	(cash2.east) -- (check2.west)
    node[midway,fill=white] [below] {Aprovação};
\end{tikzpicture}
\end{frame}

%----------------------------------------------------------------------------------------
\begin{frame}
\frametitle{Representação}
\centering
\resizebox{.8\textwidth}{!}{
\begin{tikzpicture}
\umlclass[x=0,y=0]{Purchase}{
	- value: double\\
	- closed: boolean\\
	- cashPayment: CashPayment\\
	- cardPayment: CardPayment\\
	- payPalPayment: PayPalPayment
}{
	+ finalizeCash(): boolean\\
	+ finalizeCard(): boolean\\
	+ finalizePayPal(): boolean\\
	+ setCashPayment(CashPayment c): void\\
	+ setCardPayment(CardPayment c): void\\
	+ setPayPalPayment(PayPalPayment p): void\\
	+ getTotal(): double
}
\umlclass[x=8,y=-3.5]{CashPayment}{
	- value: double\\
	- isApproved: boolean
}{
	+ execute() : void\\
	+ isApproved() : boolean
}
\umlclass[x=8,y=0]{CardPayment}{
	- value: double\\
	- isApproved: boolean
}{
	+ execute() : void\\
	+ isApproved() : boolean
}
\umlclass[x=8,y=3.5]{PayPalPayment}{
	- value: double\\
	- isApproved: boolean
}{
	+ execute() : void\\
	+ isApproved() : boolean
}
\umlunicompo[geometry=-|-,mult1=1,mult2=1,pos1=0.2,pos2=2.8]
{Purchase}{CashPayment}
\umlunicompo[geometry=-|-,mult1=1,mult2=1,pos1=0.2,pos2=2.8]
{Purchase}{CardPayment}
\umlunicompo[geometry=-|-,mult1=1,mult2=1,pos1=0.2,pos2=2.8]
{Purchase}{PayPalPayment}
\end{tikzpicture}
}
\end{frame}

%----------------------------------------------------------------------------------------
\begin{frame}[fragile]
\frametitle{Implementação}
\begin{lstlisting}
public class Purchase {
  ...

  public boolean finalize() {
    ...
    //type CashPayment
    this.cashPayment.execute();
  }
}

Purchase purchase = new Purchase();
CashPayment cash = new CashPayment(purchase.getTotal());

purchase.setCashPayment(cash);
purchase.finalizeCash();
\end{lstlisting}
\end{frame}

%----------------------------------------------------------------------------------------
\begin{frame}[fragile]
\frametitle{Implementação}
\begin{lstlisting}
public class Purchase {
  ...

  public boolean finalize() {
    ...
    //type CardPayment
    this.cardPayment.execute();
  }
}

Purchase purchase = new Purchase();
CardPayment card = new CardPayment(purchase.getTotal());

purchase.setCardPayment(card);
purchase.finalizeCard();
\end{lstlisting}
\end{frame}

%----------------------------------------------------------------------------------------
\begin{frame}[fragile]
\frametitle{Implementação}
\begin{lstlisting}
public class Purchase {
  ...

  public boolean finalize() {
    ...
    //type PayPalPayment
    this.payPalPayment.execute();
  }
}

Purchase purchase = new Purchase();
PayPalPayment payPal = new PayPalPayment(purchase.getTotal());

purchase.setPayPalPayment(payPal);
purchase.finalize();
\end{lstlisting}
\end{frame}

%----------------------------------------------------------------------------------------
\begin{frame}
\frametitle{Opção II}
\centering
\begin{tikzpicture}
\node[inner sep=0pt] (cashier0) at (-4,0)
    {
    	\href{https://www.flaticon.com/authors/freepik}
		{\includegraphics[width=.08\textwidth]{images/cashier.png}}
	};
\node[inner sep=0pt] (cash0) at (0,2)
    {
    	\href{https://www.flaticon.com/authors/dinosoftlabs}
		{\includegraphics[width=.08\textwidth]{images/cash.png}}
	};
\node[inner sep=0pt] (cash1) at (0,0)
    {
    	\href{https://www.flaticon.com/authors/smashicons}
		{\includegraphics[width=.08\textwidth]{images/credit-card.png}}
	};
\node[inner sep=0pt] (cash2) at (0,-2)
    {\href{https://www.flaticon.com/authors/roundicons}
		{\includegraphics[width=.08\textwidth]{images/paypal.png}}
	};
\node[inner sep=0pt] (check0) at (4,0)
    {
    	\href{https://www.flaticon.com/authors/maxim-basinski}
		{\includegraphics[width=.08\textwidth]{images/checked.png}}
	};
\draw[->,thick] 
	(cashier0.north east) -- (cash0.south west)
    node[midway,fill=white] {Pagamento};
\draw[->,thick] 
	(cash0.south east) -- (check0.north west)
    node[midway,fill=white] {Aprovação};
\draw[->,thick] 
	(cashier0.east) -- (cash1.west)
    node[midway,fill=white] {Pagamento};
\draw[->,thick] 
	(cash1.east) -- (check0.west)
    node[midway,fill=white] {Aprovação};
\draw[->,thick] 
	(cashier0.south east) -- (cash2.north west)
    node[midway,fill=white] {Pagamento};
\draw[->,thick] 
	(cash2.north east) -- (check0.south west)
    node[midway,fill=white] {Aprovação};
\end{tikzpicture}
\end{frame}

%----------------------------------------------------------------------------------------
\begin{frame}
\frametitle{Representação}
\resizebox{\textwidth}{!}{
\begin{tikzpicture}
\umlclass[x=0,y=0]{Purchase}{
	- value: double\\
	- closed: boolean \\
	- payment: Payment
}{
	+ finalize(): boolean\\
	+ setPayment(Payment c): void\\
	+ getTotal(): double
}
\umlclass[x=8,y=0]{Payment}{
	\# value: double\\
	\# isApproved: boolean
}{
	\umlvirt{+ execute(): void}\\
	\umlvirt{+ isApproved(): boolean}
}
\umlclass[x=3,y=-4]{CashPayment}{
}{
	+ execute() : void\\
	+ isApproved() : boolean
}
\umlclass[x=8,y=-4]{CardPayment}{
}{
	+ execute() : void\\
	+ isApproved() : boolean
}
\umlclass[x=13,y=-4]{PayPalPayment}{
}{
	+ execute() : void\\
	+ isApproved() : boolean
}

\umlunicompo[geometry=-|-,mult1=1,mult2=1,pos1=0.2,pos2=2.8]
{Purchase}{Payment}
\umlimpl{CashPayment}{Payment}
\umlimpl{CardPayment}{Payment}
\umlimpl{PayPalPayment}{Payment}
\end{tikzpicture}
}
\end{frame}

%----------------------------------------------------------------------------------------
\begin{frame}[fragile]
\frametitle{Implementação}
\begin{lstlisting}
public class Purchase {

  public boolean finalize() {
    ...
    //type Payment
    this.payment.execute();
  }
}

Purchase purchase = new Purchase();
Payment payment = new CardPayment(purchase.getTotal());
                //new CashPayment
                //new PayPalPayment

purchase.setPayment(payment);
purchase.finalize();
\end{lstlisting}
\end{frame}

%----------------------------------------------------------------------------------------
% REFERENCES
%----------------------------------------------------------------------------------------
\section*{}
\begin{frame}[allowframebreaks]
  \frametitle{Referência Bibliográfica}
  \bibliographystyle{ieeetr}
  \bibliography{references/main}
\end{frame}


\end{document} 