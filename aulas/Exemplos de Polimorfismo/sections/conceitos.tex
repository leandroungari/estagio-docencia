\section{Conceitos}

%----------------------------------------------------------------------------------------
%----------------------------------------------------------------------------------------
\subsection{Definição}

%----------------------------------------------------------------------------------------
\begin{frame}
\frametitle{Introdução}
\justifying
\quad Neste ponto da Orientação a Objetos, foram introduzidos dois pilares desse paradigma, o Encapsulamento e a Herança, enquanto o primeiro permite a implementação de componentes independentes, o segundo possibilita a reutilização hierárquica desses.
\end{frame}

%----------------------------------------------------------------------------------------
\begin{frame}
\frametitle{Polimorfismo}
\begin{block}{Definição}
\qquad Ter muitas formas. Em termos de programação, muitas formas significa que um único nome pode representar um código diferente, selecionado por algum elemento automático. Assim, o polimorfismo permite que um único nome expresse muitos comportamentos diferentes~\cite{sintes2002aprenda}.
\end{block}
\end{frame}

%----------------------------------------------------------------------------------------
%----------------------------------------------------------------------------------------
\subsection{Exemplo inicial}
%fazer exemplo de calculo de área de figuras geometricas

%----------------------------------------------------------------------------------------
\begin{frame}
\frametitle{Polimorfismo}



\end{frame}