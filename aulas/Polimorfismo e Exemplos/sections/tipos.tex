\section{Tipos de Polimorfismo}

%----------------------------------------------------------------------------------------
\begin{frame}
\centering
\Huge Tipos de Polimorfismo
\end{frame}

%----------------------------------------------------------------------------------------
%----------------------------------------------------------------------------------------
\subsection*{}

%----------------------------------------------------------------------------------------
\begin{frame}
\frametitle{Categorias}
\justifying
\quad O Polimorfismo pode ser aplicado de diversos modos, conforme a necessidade de cada implementação~\cite{sintes2002aprenda}. 
A seguir são apresentados quatros principais categorias dessa abordagem:

\begin{itemize}
	\item Polimorfismo de Inclusão
	\item Polimorfismo Paramétrico
	\item Sobreposição
	\item Sobrecarga
\end{itemize}
\end{frame}

%----------------------------------------------------------------------------------------
%----------------------------------------------------------------------------------------
\subsection{Polimorfismo de Inclusão}

%----------------------------------------------------------------------------------------
\begin{frame}
\frametitle{Definição}
\justifying
\begin{block}{Conceito}
\qquad O polimorfismo de inclusão, também conhecido como polimorfismo puro, permite que você trate objetos relacionados genericamente.
\end{block}
\end{frame}

%----------------------------------------------------------------------------------------
\begin{frame}
\frametitle{Exemplo}
\centering
\resizebox{.8\textwidth}{!}{
\begin{tikzpicture}
\umlclass[x=0,y=0]{Vehicle}{...}{...}
\umlclass[x=-3,y=-3]{Motorcycle}{}{
	+ accelerate()
}
\umlclass[x=0,y=-3]{Car}{}{
	+ accelerate()
}
\umlclass[x=3,y=-3]{Truck}{}{
	+ accelerate()
}
\umlimpl{Motorcycle}{Vehicle}
\umlimpl{Car}{Vehicle}
\umlimpl{Truck}{Vehicle}
\end{tikzpicture}
}
\end{frame}

%----------------------------------------------------------------------------------------
\begin{frame}[fragile]
\frametitle{Exemplo}
\justifying
\begin{lstlisting}
public void accelerateVehicle(Motorcycle obj){
  obj.accelerate();
}

public void accelerateVehicle(Car obj){
  obj.accelerate();
}

public void accelerateVehicle(Truck obj){
  obj.accelerate();
}
\end{lstlisting}
\end{frame}

%----------------------------------------------------------------------------------------
\begin{frame}[fragile]
\frametitle{Exemplo}
\justifying
\quad Todos os métodos anteriores são equivalentes a esse:
\\~\\
\begin{lstlisting}
public void accelerateVehicle(Vehicle obj){
  obj.accelerate();
}
\end{lstlisting}
\end{frame}


%----------------------------------------------------------------------------------------
%----------------------------------------------------------------------------------------
\subsection{Polimorfismo Paramétrico}

%----------------------------------------------------------------------------------------
\begin{frame}
\frametitle{Definição}
\justifying
\begin{block}{Conceito}
\qquad O polimorfismo paramétrico que você crie métodos e tipos genéricos. 
Essa categoria pode ser aplicada em tipos e métodos, na primeira os tipos de parâmetros e retornos podem ser generalizados, enquanto no segundo, as referências de tipos dos métodos são omitidos até o momento da execução.
\end{block}
\end{frame}

%----------------------------------------------------------------------------------------
\begin{frame}[fragile]
\frametitle{Exemplo}
\justifying
\begin{lstlisting}
//Polimorfismo parametrico de Tipo
public class Queue<T> {
	
  void enqueue(T element) { ... }

  T dequeue() { ... }

  boolean isEmpty() { ... }

}
\end{lstlisting}
\end{frame}

%----------------------------------------------------------------------------------------
\begin{frame}[fragile]
\frametitle{Exemplo}
\justifying
\begin{lstlisting}
//Metodos de adicao de elementos
int add(int a, int b) { ... }
double add(double a, double b) { ... }
int[][] add(int[][] a, int[][] b) { ... }

//Polimorfismo parametrico de Metodo
T add(T a, T b) { ... }
\end{lstlisting}
\end{frame}

%----------------------------------------------------------------------------------------
%----------------------------------------------------------------------------------------
\subsection{Sobreposição}

%----------------------------------------------------------------------------------------
\begin{frame}
\frametitle{Definição}
\justifying
\begin{block}{Conceito}
\qquad O polimorfismo de sobreposição é um tipo importante, no qual o comportamento em uma hierarquia é sobreposto por um comportamento mais específico
\end{block}
\end{frame}

%----------------------------------------------------------------------------------------
\begin{frame}[fragile]
\frametitle{Exemplo}
\justifying
\quad Uma classe~\textbf{Shape} é definida, da qual derivam outras duas classes~\textbf{Square} e~\textbf{Triangle}.
\\~\\
\begin{lstlisting}
public abstract class Shape {
	
  public abstract double area();

}
\end{lstlisting}
\end{frame}

%----------------------------------------------------------------------------------------
\begin{frame}[fragile]
\frametitle{Exemplo}
\justifying
\begin{lstlisting}
@Override
public abstract class Square extends Geometric {
	
  public double area() {
    return this.side*this.side;
  }
}

@Override
public abstract class Triangle extends Geometric {
	
  public double area() {
    return this.base*this.height/2;
  }
}
\end{lstlisting}
\end{frame}

%----------------------------------------------------------------------------------------
%----------------------------------------------------------------------------------------
\subsection{Sobrecarga}

%----------------------------------------------------------------------------------------
\begin{frame}
\frametitle{Definição}
\justifying
\begin{block}{Conceito}
\qquad O polimorfismo de sobrecarga, também conhecido como polimorfismo~\emph{ad-hoc}, permite que o nome do método seja o mesmo para diferentes métodos, diferindo apenas na quantidade de parâmetros e os tipo deles.
\end{block}
\end{frame}

%----------------------------------------------------------------------------------------
\begin{frame}[fragile]
\frametitle{Exemplo}
\justifying
\begin{lstlisting}
public class MyOwnDB {
	
  //It returns all items
  public List<Item> search() { ... } 

  //It returns all items which are filtered by condition
  public List<Item> search(String condition) { ... } 
  
  //... and ordered by attribute
  public List<Item> search(String condition, String orderBy) { ... } 
  
  //It returns limited number of items which are filtered by condition and ordered by attribute
  public List<Item> search(String condition, String orderBy) { ... } 
}
\end{lstlisting}
\end{frame}

%----------------------------------------------------------------------------------------
\begin{frame}
\frametitle{Armadilhas Polimórficas}
A utilização do polimorfismo permite muitas facilidades, porém a utilização desses recursos de modo desordenado pode incorrer em diversos problemas para a arquitetura do sistema.

\begin{itemize}
\item Mover comportamentos para cima da hierarquia\\{\footnotesize Mover métodos acima pode dar capacidades a classes em que estas não deveriam ter.}

\end{itemize}
\end{frame}

%----------------------------------------------------------------------------------------
\begin{frame}
\frametitle{Armadilhas Polimórficas}

\begin{itemize}
\item Sobrecarga de desempenho\\{\footnotesize A generalização de tipos de parâmetros e retorno de métodos traz perda de desempenho em tempo de execução, em que a verificação de tipos, que era feita estaticamente, será feita dinamicamente, demandando processamento.}
\item Perda de comportamento\\{\footnotesize Na generalização de tipos específicos, um método ou atributo pertencente a uma subclasse se torna inacessível.}
\end{itemize}

\end{frame}